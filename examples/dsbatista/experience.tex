%-------------------------------------------------------------------------------
%	SECTION TITLE
%-------------------------------------------------------------------------------
\cvsection{Professional Experience}


%-------------------------------------------------------------------------------
%	CONTENT
%-------------------------------------------------------------------------------
\begin{cventries}
  \cventry
    {Senior Data Scientist} % Job title
    {Veeva Systems} % Organization
    {Berlin, Germany} % Location
    {March 2023 - Present } % Date(s)
	{
      \begin{cvitems} % Description(s) of tasks/responsibilities
		\item {Developed a supervised model to categorise organisations in the life-sciences domain using an extensive multi-lingual dataset of 2 million samples which reduced the workload on the curators by 10\%, }
		\item {Improved a model to establish associations between researchers to their respective activities by relying on contextual embeddings applied to text extracted from different activities from 3.2 billion training samples.}
		\item {Both models were developed with the PySpark Machine Learning library and AWS EMR computing cluster.}
		\item {Introduced Python best practices: tests and coverage, code linting, PEP 8 code style, and type hints all integrated into a CI/CD pipeline.}
      \end{cvitems}
    }

%---------------------------------------------------------
  \cventry
    {Lead NLP Engineer} % Job title
    {TripActions/Navan (acquired Comtravo GmbH)} % Organization
    {Berlin, Germany} % Location
    {May 2022 - Feb 2023} % Date(s)
	{
      \begin{cvitems} % Description(s) of tasks/responsibilities
		% \item {Defined an intent and entities schema and set up the annotation infrastructure based on the open-source package argilla.io}
		% \item {Defined and implemented an entropy-based active-learning strategy to select samples to annotate.}
		% \item {Conducted onboarding sessions, introducing annotators to the annotation task.}
		% \item {Fine-tuned Transformer models for intent detection and entity recognition, and performed threshold tuning for deployment in production.}
		\item {My team was tasked with developing a chatbot's NLP/NLU component to automate customer support.}
		\item {Defined an intent and entities schema and onboarded annotators to the annotation task.}
		\item {Set up the infrastructure for annotation based on argilla.io open source annotation tool.}
		\item {Defined and implemented an entropy-based active-learning strategy to select samples to annotate.}
		\item {Fine-tuned, evaluated and deployed different classifiers based on Transformer architectures.}
      \end{cvitems}
    }


%---------------------------------------------------------

\cventry
    {Lead NLP Engineer} 
    {Comtravo GmbH} 
    {Berlin, Germany} 
    {Aug. 2019 - May 2021}
    {
      \begin{cvitems} 
		  \item {Led a team of 3 developers + 4 annotators, guiding them in technical decisions and supervised system changes.}
		  \item {Managed bi-weekly sprint planning and execution, coordinating development tasks based on system performance and feature requests.}
		  \item {Defined quantifiable measures in collaboration with the Data Engineering team, resulting in performance reports and monitoring dashboards.}
		  \item {Supervised the data annotation: ensuring quality and annotation consistency across corpora.}
		  \item {Maintained an active role in software development.}
        \end{cvitems}
 	}


  \cventry
    {Senior NLP Engineer}
    {} % Organization EMPTY
    {} % Location EMPTY
    {Aug. 2017 - Jul. 2019} % Date(s)
    {
      \begin{cvitems}
		\item {Participated in building from scratch the system which automatically answers incoming email travel requests.}
		\item {Developed text classifiers using different architectures.}
		\item {Developed sequence tagging classifiers for fine-grained NER.}
		\item {Built airport and train stations Knowledge Bases based on open resources: Wikidata, GeoNames, DB open-data and in-house operational data.}
		\item {Developed entity-linking approaches to map recognized entities into unique Knowledge Base identifiers, e.g: airports, train stations, hotels, geographic locations.}
		\item {Developed several modularised Python components with type-annotations, linting, test coverage of 95\%.}
        \end{cvitems}
 	}

%------------------------------------------------------------------------------------------------------------



\begin{comment}
  \cventry
    {Lead NLP Engineer} % Job title
    {Comtravo GmbH} % Organization
    {Berlin, Germany} % Location
    {Aug. 2017 - Apr. 2022} % Date(s)
	{
      \begin{cvitems} % Description(s) of tasks/responsibilities
		\item {Joined as an NLP Engineer in August 2017, was promoted to Senior NLP Engineer in July 2019, and in June 2021 to Lead NLP Engineer.}
        \item {Led a team of 3 developers + 4 annotators, working on the system that automatically answers incoming email travel requests and assists travel agents in handling them. Coordinating development tasks based on system performance and feature requests.}
        \item {Trained, evaluated and improved different models for text classification and fine-grained NER, increasing the performance of identifying specific booking requests and performing information extraction to automatically fulfil booking requests.}
		\item {Developed algorithms to map input text into unique Knowledge Base identifiers, e.g: airports, train stations, hotels, geographic locations.}		
      \end{cvitems}
    }
\end{comment}




%---------------------------------------------------------
  \cventry
    {Data Engineer} % Job title
    {HelloFresh SE} % Organization
    {Berlin, Germany} % Location
    {Jan. 2016 - Jun. 2017} % Date(s)
    {
      \begin{cvitems} % Description(s) of tasks/responsibilities
        \item {Built and maintained several ETLs using PySpark (Apache Spark) and Hive.}
        \item {Built a prototype to manage ETLs pipelines based on Airflow Operators which later become the ETL management platform in production.}
        \item {Built a classifier using NLTK and scikit-learn linear models to identify customer review mentions to different types of issues with the meal kits.}
      \end{cvitems}
    }
	
%---------------------------------------------------------
  \cventry
    {Researcher and Developer: \href{http://arquivo.pt/wayback/20151118124735/http://dmir.inesc-id.pt/project/Reaction}{REACTION - Computational Journalism}} % Job title
    {INESC-ID Research \& Development Institute} % Organization % - \href{http://arquivo.pt/wayback/20151118124735/http://dmir.inesc-id.pt/project/Reaction}{REACTION - Computational Journalism
    {Lisbon, Portugal} % Location
    {Jun. 2011 - Apr. 2014} % Date(s)
    {
      \begin{cvitems} % Description(s) of tasks/responsibilities
        \item {Explored and implemented methods for entity-relationship extraction, based on: hand-built patterns, supervised linear classifiers with linguistic features and semi-supervised methods based on seed relationships and large amounts of unannotated text.}
        \item {Developed a method to link personalities in news articles to Wikipedia entries based on textual similarities and Wikipedia graph structure.}
        \item {Built a graph based on topic models extracted from news articles and person co-occurrences, allowing to explore topics connecting persons.}
      \end{cvitems}
    }


%---------------------------------------------------------
  \cventry
    {Researcher and Developer: \href{https://arquivo.pt/wayback/20181014115718/http://xldb.di.fc.ul.pt/wiki/Grease}{GREASE - Geographic Reasoning for Search Engines}} % Job title
    {LaSIGE - Department of Informatics Research Unit } % Organization %
    {Lisbon, Portugal} % Location
    {Sep. 2008 - Oct. 2010} % Date(s)
    {
      \begin{cvitems} % Description(s) of tasks/responsibilities
        \item {Developed a method to disambiguate toponyms based on textual context, information content and topological similarity measures.}
        \item {Built an alignment method between two geo-ontologies resulting in a single linked-data ontology, allowing the inclusion of features from both ontologies in SPARQL queries.}
		% \item {Identified, based on n-grams, the language of each page of a 10 million pages web crawl and used this data to identify top Portuguese names overlapping with toponyms}. 
		% \item {Published scientific articles with the outcome of these tasks.}
      \end{cvitems}
    }

%---------------------------------------------------------
  \cventry
    {Software Developer} % Job title
    {Nokia-Siemens Networks} % Organization % - \href{https://arquivo.pt/wayback/20181014115718/http://xldb.di.fc.ul.pt/wiki/Grease}{Geographic Reasoning for Search Engines}
    {Lisbon, Portugal} % Location
    {Oct. 2007 - Jul. 2008} % Date(s)
    {
      \begin{cvitems} % Description(s) of tasks/responsibilities
        \item {Developed data collection modules for a GSM monitoring system using Java, CORBA Architecture and Oracle RDBMS}
      \end{cvitems}
    }

\end{cventries}
