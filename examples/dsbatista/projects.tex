\cvsection{Personal Projects}

\begin{cventries}
  \cventry
    {Analysis of support and opposition political relationships} % Degree
	{Politiquices.PT - https://www.politiquices.pt} % Institution
    {} % Location
    {} % Date(s)
    {
      \begin{cvitems} % Description(s) bullet points
		  \item {An award-winning project that explores political interactions of support and opposition based on news articles headlines. I developed supervised models to detect political interactions and link politicians to Wikidata, generating an RDF semantic graph. Users can query the graph via a SPARQL endpoint or explore it interactively through the website. The project has attracted interest from journalists, political scientists, and social humanities researchers.}		  
        %\item {A website which allows to explore political interactions of support and opposition based on news articles, which was awarded the \textbf{2nd place on the "Arquivo.pt Awards 2021"} organised by the Portuguese Web Archive.}
		%\item {I've built and tuned supervised models to detect relationships between politicians and link them to Wikidata, based on news headlines. The trained models were applied to archived news articles, generating an RDF semantic graph connecting politicians through a support or opposition relationship sustained by news articles.}
		%\item {Using a SPARQL endpoint it's possible to issue queries like: \textit{Which politicians affiliated with party X opposed/supported politicians from party Y?}}
		% \item {The interface to the SPARQL endpoint is based on FastAPI and ReactJS.}
		% \item {The project gained the interest of journalists, political scientists and social humanities researchers.}
      \end{cvitems}
    }
\end{cventries}

\begin{cventries}
 \cventry
   {Bootstrapping Semantic Relationships with Distributional Semantics} % Degree
   {BREDS - https://pypi.org/project/breds} % Institution
   {} % Location
   {} % Date(s)
   {
     \begin{cvitems} % Description(s) bullet points
		\item {An open-source Python package stemming from my Ph.D. thesis, designed to extract named-entity relationships without labeled data. It utilizes an initial seed pairs of named-entities to learn extraction patterns, expanding and generalising relationships through distributional semantics while minimising semantic drift.}
     \end{cvitems}
   }
\end{cventries}

\begin{cventries}
 \cventry
   {Named-Entity Recognition considering partial matching} % Degree
   {nervaluate - https://pypi.org/project/nervaluate} % Institution
   {} % Location
   {} % Date(s)
   {
     \begin{cvitems} % Description(s) bullet points
     	\item {An open-source Python package for evaluating named-entity recognition systems considering partial entity matching, this started with a blog post showing a proof-of-concept which then gathered interest and contributions from multiple collaborators.}
     \end{cvitems}
   }
\end{cventries}


