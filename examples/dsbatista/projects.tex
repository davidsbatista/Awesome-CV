\cvsection{Personal Projects}

% <position>}{<title>}{<location>}{<date>}{<description>}

\begin{cventries}
  \cvproject
    {} % position
	{Politiquices.PT - Extraction of support and opposition political relationships} % title
    {} % location
    {Sep 2020 - Apr 2021} % date(s)
    {
      \begin{cvitems} % Description(s) bullet points		  
		  \item{An award-winning project (2nd place, "Arquivo.pt Awards 2021") which extracts political interactions from Portuguese news headlines, it uses supervised models to detect interactions, linking the involved politicians to Wikidata entities creating an enriched RDF semantic graph. The graph allows for exploration by revealing nodes or graph clusters that depict the evolving political relationships of opposition or support between specific personalities, supported by information from news articles. I developed it during the 2020 pandemic while on short-time work, bringing to life an idea conceived during my Ph.D. - \url{https://www.politiquices.pt}}
      \end{cvitems}
    }
\end{cventries}




\begin{cventries}
 \cvproject
   {} % Institution
   {Bootstrapping Semantic Relationships with Distributional Semantics} % Degree
   {} % Location
   {Apr 2015 - Sep 2015} % Date(s)
   {
     \begin{cvitems} % Description(s) bullet points
		\item {\texttt{breds} is an open-source Python package stemming from my Ph.D. thesis, designed to extract named-entity relationships without labeled data. It utilizes an initial seed pairs of named-entities to learn extraction patterns, expanding and generalising relationships through distributional semantics while minimising semantic drift. - \url{https://pypi.org/project/breds}}
     \end{cvitems}
   }
\end{cventries}

\begin{cventries}
 \cvproject
   {} % Institution
   {Named-Entity Recognition considering partial matching} % Degree
   {} % Location
   {May 2018 - Sep 2018} % Date(s)
   {
     \begin{cvitems} % Description(s) bullet points
     	\item {\texttt{nervaluate} is an open-source Python package that evaluates named-entity recognition systems, including partial entity matching. It began as a proof-of-concept I worked on during my time at Comtravo and has since evolved into a mature Python package with contributions from various collaborators. - \url{https://pypi.org/project/nervaluate}}
     \end{cvitems}
   }
\end{cventries}
